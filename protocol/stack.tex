\documentclass[border=2mm]{standalone}

\usepackage{hyperref}  % \href
\usepackage{textcomp}  % \textrightarrow
\usepackage{pgf}
\usepackage{tikz}

\setlength{\parindent}{0pt}
\thispagestyle{empty}

\usetikzlibrary{backgrounds}  % background-layer library
\usetikzlibrary{positioning}  % node-placement library
\usetikzlibrary{shapes.misc}  % for 'rounded rectangle'

\hypersetup{colorlinks}
\hypersetup{urlcolor=black}
\hypersetup{filecolor=red}  % \href{non-URL-placeholder}{text}

% Helper for code-like font styling
\newcommand{\code}{\small\ttfamily}

% \method{name}{arguments}{results}
%
% If <results> is non-empty, formatted as:
%
%   <name>(<arguments>) -> <results>
%
% otherwise formatted as:
%
%   <name>(<arguments>)
\newcommand{\method}[3]{
  \code\color{gray}
  #1(#2)
  \ifx\\#3\\
  \else
    \textrightarrow\ #3
  \fi
}

% \event{text}
%
% An event that can trigger notifiee callbacks
\newcommand{\event}[1]{\code\color{brown}#1}

% \target{text}{node-name}
%
% Target text for a TikZ line
\newcommand{\target}[2]{\tikz[baseline] \node[target] (#2) {#1};}

\pgfdeclarelayer{background}
\pgfdeclarelayer{foreground}
\pgfsetlayers{background,main,foreground}

\tikzstyle{data}=[
  rounded rectangle,
  draw=black!50,
  fill=black!5,
  thick,
]

\tikzstyle{protocol}=[
  rounded rectangle,
  draw=blue!50,
  fill=blue!5,
  thick,
]

\tikzstyle{interface}=[
  rectangle,
  draw=green!50,
  fill=green!5,
  thick,
  align=flush left
]

\tikzstyle{actor}=[
  rectangle,
  draw=brown!50,
  fill=brown!5,
  thick,
  align=flush left
]

\tikzstyle{module}=[
  rounded rectangle,
  draw=black!50,
  dashed,
]

\tikzstyle{module-label}=[
  rounded rectangle,
]

\tikzstyle{merkle-link}=[
  draw=blue
]

\tikzstyle{extend}=[
  draw=red!50!black
]

\tikzstyle{process}=[
  draw=red
]

\tikzstyle{notify}=[
  draw=brown
]

\tikzstyle{use}=[
  draw=green!50!black
]

\tikzstyle{target}=[
  inner sep=0,
  text=green!50!black,
  anchor=text
]

\tikzstyle{every picture}+=[remember picture]  % links between pictures

\begin{document}
\begin{tikzpicture}
  \node[data] (merkle) {
    \href{https://github.com/ipfs/specs/tree/master/protocol#merkledag----making-sense-of-data}{
      Merkle object
    }
    % Also in https://github.com/ipfs/specs/pull/7
  };

  \node[data] (key) [above of=merkle] {
    \href{https://github.com/ipfs/specs/pull/7}{Key}
    % Also in https://github.com/ipfs/specs/tree/master/keystore#structures
  };
  \draw[extend] (merkle) -- (key);
  \node[data] (signature) [above=of key] {
    \href{https://github.com/ipfs/specs/pull/7}{Signature}
    % https://github.com/ipfs/specs/tree/master/keystore#structures
  };
  \draw[extend,bend right=45] (merkle) to (signature);
  \draw[merkle-link,->] (signature) -- (key);

  \node[data,align=left] (record) [right=of key] {
    \href{https://github.com/ipfs/specs/pull/7}{Record} \\
    \method{Compare}{other}{\{-1|0|1\}}
  };
  \draw[extend,bend right=5] (merkle) to (record);

  \node[data] (layout) [left=1.5 of key] {
    \href{https://github.com/ipfs/specs/issues/10}{Layout}
  };
  \draw[extend] (merkle) -- (layout);
  \node[data] (chunk) [left=of layout] {
    \href{Chunking spec?}{Chunk}
  };
  \draw[extend,bend left=5] (merkle) to (chunk);

  \node[data] (file) [above=of chunk] {
    \href{https://github.com/ipfs/specs/tree/master/protocol#unixfs----representing-traditional-files}{File}
  };
  \draw[extend] (chunk) -- (file);
  \node[data] (directory) [above=of layout] {
    \href{https://github.com/ipfs/specs/tree/master/protocol#unixfs----representing-traditional-files}{Directory}
  };
  \draw[extend] (layout) -- (directory);

  \node[data] (block) [right=6 of merkle] {
    \href{https://gist.github.com/jbenet/d1fedddfef85f0c4efd5}{Block}
  };
  \draw[process,<->] (merkle) -- (block);

  \node[protocol] (multihash) [above=of block] {
    \href{https://github.com/jbenet/multihash/}{Multihash}
  };
  \draw[process,->] (block) -- (multihash);

  \node[interface] (datastore) [right=of multihash] {
    \href{https://gist.github.com/jbenet/d1fedddfef85f0c4efd5}{Datastore} \\
    \method{Put}{\target{Key}{datastore-key}, \target{Value}{datastore-value}}{} \\
    \method{Get}{Key}{Value} \\
    \method{Has}{Key}{bool} \\
    \method{Delete}{Key}{}
  };
  \draw[use,bend right] (multihash) to (datastore-key);
  \draw[use,bend right] (block) to (datastore-value);

  \node[interface] (blockstore) [above=of datastore] {
    \href{https://gist.github.com/jbenet/d1fedddfef85f0c4efd5}{Blockstore} \\
    \method{AllKeys}{}{KeyIterator}
  };
  \draw[extend] (datastore) -- (blockstore);

  \node[interface] (notifier) [right=0.3 of datastore] {
    \href{https://gist.github.com/jbenet/d1fedddfef85f0c4efd5}{Notifier} \\
    \method{Notifiees}{}{NotifieeIterator} \\
    \method{AddNotifiee}{Notifiee}{} \\
    \method{RemoveNotifiee}{Notifiee}{}
  };

  \node[actor] (exchange-server) [above=of blockstore] {
    \href{https://gist.github.com/jbenet/d1fedddfef85f0c4efd5}{Exchange server} \\
    \event{BlockPut} \\
    \event{NewBlockPut} \\
    \event{BlockDeleted}
  };
  \draw[extend] (blockstore) -- (exchange-server);
  \draw[extend,bend right] (notifier) to (exchange-server);

  \node[interface] (exchange-client) [above right=of exchange-server] {
    \href{https://gist.github.com/jbenet/d1fedddfef85f0c4efd5}{Exchange client} \\
    \method{GetBlock}{Key}{Block}
  };
  \draw[use] (exchange-server) -- (exchange-client);

  \node[interface] (recordstore) [above=3 of record] {
    \href{https://github.com/ipfs/specs/pull/7}{Record store (distributed)} \\
    \method{Put}{Key, \target{Value}{recordstore-value}}{} \\
    \method{Get}{Context, Key}{ValueIterator}
  };
  \draw[use] (record) to [out=140,in=-90] (signature.east) to [out=90,in=-120] (recordstore-value);

  \node[interface] (ipns) [above=of recordstore] {
    \href{https://github.com/ipfs/specs/tree/master/protocol#naming----pki-namespace-and-mutable-pointers}{IPNS} \\
    \method{Publish}{Key, Link}{} \\
    \method{Resolve}{Key}{}
  };
  \draw[use] (recordstore) -- (ipns);

  \node[interface] (provider-table) [above left=0.5 of exchange-server] {
    \href{TODO: Providers spec}{Provider table} \\
    \small and routing
  };
  \draw[use] (recordstore) -- (provider-table);
  \draw[notify] (exchange-server) -- (provider-table);
  \draw[use,bend left=10] (provider-table) to (exchange-client);

  \node[interface] (ipfs) [above=2 of exchange-server] {
    \href{https://github.com/ipfs/specs/tree/master/protocol#naming----pki-namespace-and-mutable-pointers}{IPFS}
  };
  \draw[use] (exchange-server) -- (ipfs);
  \draw[use] (exchange-client) -- (ipfs);
  \draw[use] (provider-table) -- (ipfs);

  \begin{pgfonlayer}{background}
    \path (file.west)+(-0.5,0.6) node (unixfs-a) {};
    \path (directory.east)+(+0.3,-0.6) node (unixfs-b) {};
    \path[module,fill=yellow!5] (unixfs-a) rectangle (unixfs-b);
    \node[module-label,fill=yellow!5] (unixfs) [right=3mm of unixfs-a] {
      \href{https://github.com/ipfs/specs/tree/master/protocol#unixfs----representing-traditional-files}{Unix filesystem}
    };

    \path (signature.west)+(-0.3,0.6) node (crypto-a) {};
    \path (signature.east)+(+0.3,-2) node (crypto-b) {};
    \path[module,fill=orange!5] (crypto-a) rectangle (crypto-b);
    \node[module-label,fill=orange!5] (crypto) [above=0.2 of signature] {
      \href{https://github.com/ipfs/specs/tree/master/keystore}{Cryptography}
    };
  \end{pgfonlayer}
\end{tikzpicture}
\end{document}
